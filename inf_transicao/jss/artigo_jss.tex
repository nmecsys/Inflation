\documentclass[article]{jss}
\usepackage[utf8]{inputenc}

\providecommand{\tightlist}{%
  \setlength{\itemsep}{0pt}\setlength{\parskip}{0pt}}

\author{
Fernando Teixeira\\Getulio Vargas Foundation \And Second Author\\Affiliation
}
\title{Triple Filter Core Inflation: measuring inflation trajectory with
\pkg{INFLATION}}

\Plainauthor{Fernando Teixeira, Second Author}
\Plaintitle{Triple Filter Core Inflation: measuring inflation trajectory with
Package INFLATION}
\Shorttitle{\pkg{foo}: A Capitalized Title}

\Abstract{
The abstract of the article.
}

\Keywords{keywords, not capitalized, \proglang{R}}
\Plainkeywords{keywords, not capitalized, R}

%% publication information
%% \Volume{50}
%% \Issue{9}
%% \Month{June}
%% \Year{2012}
%% \Submitdate{}
%% \Acceptdate{2012-06-04}

\Address{
    Fernando Teixeira\\
  Getulio Vargas Foundation\\
  First line Second line\\
  E-mail: \email{fernando.teixeira@fgv.br}\\
  URL: \url{http://rstudio.com}\\~\\
    }

\usepackage{amsmath}

\begin{document}

\section{Introduction}\label{introduction}

This template demonstrates some of the basic latex you'll need to know
to create a JSS article.

\subsection{Code formatting}\label{code-formatting}

Don't use markdown, instead use the more precise latex commands:

\begin{itemize}
\tightlist
\item
  \proglang{Java}
\item
  \pkg{plyr}
\item
  \code{print("abc")}
\end{itemize}

\section{R code}\label{r-code}

Can be inserted in regular R markdown blocks.

\begin{CodeChunk}

\begin{CodeInput}
R> x <- 1:10
R> x
\end{CodeInput}

\begin{CodeOutput}
 [1]  1  2  3  4  5  6  7  8  9 10
\end{CodeOutput}
\end{CodeChunk}



\end{document}

