\documentclass[article]{jss}
\usepackage[utf8]{inputenc}

\providecommand{\tightlist}{%
  \setlength{\itemsep}{0pt}\setlength{\parskip}{0pt}}

\author{
Pedro Costa Ferreira\\FGV \And Fernando Teixeira\\FGV \And Talitha Speranza\\FGV
}
\title{Triple Filter: measuring trajectory with \pkg{INFLATION}}

\Plainauthor{Pedro Costa Ferreira, Fernando Teixeira, Talitha Speranza}
\Plaintitle{Triple Filter Core Inflation: measuring trajectory with Package
INFLATION}
\Shorttitle{\pkg{INFLATION}: Triple Filter Core Inflation}

\Abstract{
The \proglang{R} package \pkg{INFLATION} provides functions to estimate
the core inflation. Apart from the established exclusion
(\code{core.ex}), trimming (\code{core.ma}) and double weighting
(\code{core.dp}) core inflation filters the package brings a new way to
estimate core inflation. In countries with higher inflation rates than
traditional OECD countries, as is the case of Brazil, the existent
filters do not seem to deliver much information about prices' level. To
address this issue, we implement the Triple-Filter core inflation. This
method consists of trimming the mean with smoothed items, perform a
seasonal adjustment and finally applying moving average. This paper
constitutes a companion paper to the package, introducing the core
functions parameters, detailing all the core estimation techniques and
providing implementation details. It also presents examples where the
Triple Filter outperforms the more orthodox alternatives.
}

\Keywords{inflation, core, filter, \proglang{R}}
\Plainkeywords{inflation, core, filter, R}

%% publication information
%% \Volume{50}
%% \Issue{9}
%% \Month{June}
%% \Year{2012}
%% \Submitdate{}
%% \Acceptdate{2012-06-04}

\Address{
    Pedro Costa Ferreira\\
  FGV\\
  Endereço: Rua Barão de Itambi, 60 - Botafogo CEP: 22231-000\\
  E-mail: \email{pedro.ferreira@fgv.br}\\
  URL: \url{http://portalibre.fgv.br/}\\~\\
      Fernando Teixeira\\
  FGV\\
  Endereço: Rua Barão de Itambi, 60 - Botafogo CEP: 22231-000\\
  E-mail: \email{fernando.teixeira@fgv.br}\\
  URL: \url{http://portalibre.fgv.br/}\\~\\
      Talitha Speranza\\
  FGV\\
  Endereço: Rua Barão de Itambi, 60 - Botafogo CEP: 22231-000\\
  E-mail: \email{talitha.speranza@fgv.br}\\
  URL: \url{http://portalibre.fgv.br/}\\~\\
  }

\usepackage{amsmath}

\begin{document}

\section{Introduction}\label{introduction}

The core inflation measures are used by monetary authorities as a tool
to measure the stabilization of prices in the economy. Despite being a
popular term among policymakers, there is still no consensus on its
definition nor on what it plans to capture. The consensus is that the
change in the price level, despite being a monetary phenomenon, can be
influenced also by non-monetary events such as, for example, bad weather
conditions that make food prices more expensive because of a reduced
supply of these products to the population. However, since this event is
temporary, with an improving climate food prices may fall again. This
transient behavior thus adds noise to the inflation rate and, therefore,
the monetary authorities should be able to distinguish between a
transient effect and a persistent effect on the price level when making
their decisions. Given this, an inflation measure free of such
interference is desirable.

This paper presents the R \citep{rproj} package INFLATION \citep{inf}
which provides automatic and flexible methods to compute core inflation
based on input given. The package presents three traditional ways of
calculating the inflation's core and an additional one created by
\citep{triple}

\section{Exclusion Core Inflation}\label{exclusion-core-inflation}

The exclusion core inflation is easy to understand and doesn't demand
much computational power which makes it a big asset. With this method we
perform an exclusion of some of the prices' index items.

\subsection{Compute time series' standard
deviation}\label{compute-time-series-standard-deviation}

\begin{align}
\sigma_{i} = \sqrt{\frac{\sum\nolimits_{t=1}^{T}{(\pi_{i,t}-\bar{\pi}_{i}})^2}{T-1}}
\end{align}

Where:

\begin{itemize}
\tightlist
\item
  \(\sigma_i\) is item's \(i\) standard deviation
\item
  \(\pi_{i,t}\) is the item's \(i\) variation in time \(t\) ;
\item
  \(\bar{\pi}_{i} = \frac{\sum_{t=1}^{T}{\pi_{i,t}}}{T}\)
\end{itemize}

Don't use markdown, instead use the more precise latex commands:

\begin{itemize}
\tightlist
\item
  \proglang{Java}
\item
  \pkg{plyr}
\item
  \code{print("abc")}
\end{itemize}

\section{R code}\label{r-code}

Can be inserted in regular R markdown blocks.

\begin{CodeChunk}

\begin{CodeInput}
R> x <- 1:10
R> x
\end{CodeInput}

\begin{CodeOutput}
 [1]  1  2  3  4  5  6  7  8  9 10
\end{CodeOutput}
\end{CodeChunk}

\bibliography{artigo_jss}


\end{document}

